\documentclass[11pt]{article}
\usepackage{fontspec, xunicode, xltxtra}
\usepackage[tmargin=1in,bmargin=1in,lmargin=1.25in,rmargin=1.25in]{geometry}
\usepackage{ctex}
\usepackage{xcolor}
\pagecolor[RGB]{7,54,66}
\definecolor{fontcolor}{RGB}{253,246,227}
\begin{document}
\color{fontcolor}{}
传感器检测技术及仪表
\tableofcontents
\section{绪论}

\section{检测系统的基本特性}

\section{阻抗型传感器}
\subsection{电阻型传感器}
\subsubsection{电位器式传感器}%
\label{ssub:dian_wei_qi_shi_chuan_gan_qi_}

\subsubsection{应变式传感器}%
\label{ssub:ying_bian_shi_chuan_gan_qi_}

\subsubsection{热电阻和热敏电阻}%
\label{ssub:re_dian_zu_he_re_min_dian_zu_}

\subsection{电容型传感器}
\subsubsection{变极距型电容式传感器}%
\label{ssub:bian_ji_ju_xing_dian_rong_shi_chuan_gan_qi_}

\subsubsection{变面积型电容式传感器}%
\label{ssub:bian_mian_ji_xing_dian_rong_shi_chuan_gan_qi_}

\subsubsection{变介质型电容式传感器}%
\label{ssub:bian_jie_zhi_xing_dian_rong_shi_chuan_gan_qi_}

\subsubsection{测量电路}%
\label{ssub:ce_liang_dian_lu_}

\paragraph{比例运算法电路}%
\label{par:bi_li_yun_suan_fa_dian_lu_}

$$
U_0 = -U_i \frac{C_0}{C_x} 
$$

\paragraph{交流(变压器)电桥}%
\label{par:jiao_liu_bian_ya_qi_dian_qiao_}

$$
   \dot U_0 = \frac{\dot U}{2} \cdot \frac{C_1 - C_2}{C_1 + C_2} 
$$

对变极距型差动电容传感器,有
$$
   \dot U_0 = \frac{\dot U}{2} \cdot \frac{\Delta d}{d_0} 
$$
可见对于\bold{变极距型电容式传感器},在电阻极大时呈线性。

\subsubsection{电容式传感器及其应用}%
\label{ssub:dian_rong_shi_chuan_gan_qi_ji_qi_ying_yong_}

\subsubsection{电容式传感器与智能手机}%
\label{ssub:dian_rong_shi_chuan_gan_qi_yu_zhi_neng_shou_ji_}

\subsubsection{知识讲座 --- 生物识别技术}%
\label{ssub:subsubsection_namezhi_shi_jiang_zuo_sheng_wu_shi_bie_ji_zhu_}

通过对各种生物特征进行识别,指纹、虹膜

\paragraph{生物识别的技术核心}%
\label{par:sheng_wu_shi_bie_de_ji_zhu_he_xin_}

生物识别的技术核心在于将各种生物特征转换成数字信息,并使用有效的匹配算法来进行鉴别

略
\end{document}
