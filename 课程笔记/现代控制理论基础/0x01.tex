\documentclass[11pt]{book}
\usepackage{xcolor}
\pagecolor[RGB]{199, 237, 204}
\usepackage{fontspec, xunicode, xltxtra}
\usepackage[tmargin=1in,bmargin=1in,lmargin=1.25in,rmargin=1.25in]{geometry}
\usepackage{ctex}

\begin{document}

\title{现代控制理论基础课程笔记}
\author{DX}
\date{2019-09-17}
\maketitle

\chapter{控制系统的状态空间描述}   

\section{控制系统状态空间的基本概念}


\section{由系统的物理模型建立状态空间表达式}


\section{根据微分方程建立状态空间表达式}

\paragraph{微分方程中不含输入函数导数项}%
\label{par:微分方程中不含输入函数导数项}

输入函数中不含导入项时系统微分方程:

$$
y^{(n)} + a_{n-1}y^{(n-1)} + a_{n-2}y^{(n-2)} + \ldots + a_1 \dot y + a_0y = \beta_0u
$$

$$
\ldots
$$

\subparagraph{能控规范型}%
\label{par:能控规范型}

友矩阵

$$
A =
\left[
\begin{matrix}
0	1	0		\cdots	x \\
0	0	1		\cdots	x \\
\vdots	\vdots	\vdots	\ddots	\vdots \\
0	0	0	\cdots	1
\end{matrix}
\right]
$$

\paragraph{微分方程中含输入函数导数项}%
\label{par:微分方程中含输入函数导数项}


\section{系统传递函数矩阵与状态空间表达式的相互转换}

\paragraph{系统传递函数}%
\label{par:系统传递函数}

\subsection{直接分解法}%

$\ldots$

\subsubsection{根据传递函数的分布情况建立系统状态空间表达式}

\paragraph{传递函数极点互异}%
\label{par:传递函数极点互异}

\subsection{状态空间表达式的线性变换及规范化}

\subsubsection{系统的线性等价变换}%
\label{ssub:xi_tong_de_xian_xing_deng_jie_bian_huan_}

\subsubsection{系统的特征值及其不变性}%
\label{ssub:xi_tong_de_te_zheng_zhi_ji_qi_bu_bian_xing_}

\subsubsection{将系数矩阵 $A$ 化为对角线规范型}%
\label{ssub:jiang_xi_shu_ju_zhen_a_hua_wei_dui_jiao_xian_gui_fan_xing_}

\subsubsection{将系数矩阵 $A$ 化为约当(jordan)规范型}%
\label{ssub:jiang_xi_shu_ju_zhen_a_hua_wei_yue_dang_jordan_gui_fan_xing_}

特征值的代数重数等于几何重数时才能化为对角线规范型,否则只能化为 jordan 规范型,即:
$$
	\left\{
	   \begin{array}
		   \overline{\dot{x}} = \overline{Ax} + \overline{Bu} \\
		   y = \overline{Cx} + \overline{Du}
	   \end{array}
	\right.
$$
其中,
$$
\overline{B} = P^{-1} B, \overline{C} = CP, D = D, \overline{A} = P^{-1}AP = 
\left[
	\begin{array}{cccc}
		J_1 & & &  0 \\
			& \lambda_{m+1} & & \\
			& & \ddots &	\\
			0 & & & \lambda_n
	\end{array}
\right]
= \mathbf{J}
$$

\subsection{从状态空间表达式求取传递函数矩阵}

对线性定长系统的状态空间方程求取拉氏变换,可得:
$$
aX(s) - X(0) = AX(s) + BU(s) \\
Y(s) = CX(s) + DU(s)
$$
设初始条件 $X(0) = 0$ ,可得:
$$
sX(s) = AX(s) + BU(s) \\
Y(s) = CX(s) + DU(s) 
$$
可求得系统的传递函数为:
$$
G(s) = \frac{Y(s)}{U(s)} = C (sI - A)^{-1} B + D	
$$

\subsection{离散系统的状态空间表达式}

\section{小结}

\begin{enumerate}
	\item 状态空间表达式是由\textbf{输入方程}和\textbf{输出方程}组成的
	\item \cdots
\end{enumerate}

\end{document}
