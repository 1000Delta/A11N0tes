\documentclass[11pt]{article}
\usepackage{xcolor}
\pagecolor[RGB]{199, 237, 204}
\usepackage{fontspec, xunicode, xltxtra}
\usepackage[tmargin=1in,bmargin=1in,lmargin=1.25in,rmargin=1.25in]{geometry}

\begin{document}

\title{现代控制理论基础课程笔记}
\author{DX}
\date{2019-09-17}
\maketitle

\section{控制系统的状态空间描述}   

\subsection{控制系统状态空间的基本概念}


\subsection{由系统的物理模型建立状态空间表达式}


\subsection{根据微分方程建立状态空间表达式}

\paragraph{微分方程中不含输入函数导数项}%
\label{par:微分方程中不含输入函数导数项}

输入函数中不含导入项时系统微分方程:

$$
y^{(n)} + a_{n-1}y^{(n-1)} + a_{n-2}y^{(n-2)} + \ldots + a_1 \dot y + a_0y = \beta_0u
$$

$$
\ldots
$$

\subparagraph{能控规范型}%
\label{par:能控规范型}

友矩阵

$$
A =
\left[
\begin{matrix}
0	1	0		\cdots	x \\
0	0	1		\cdots	x \\
\vdots	\vdots	\vdots	\ddots	\vdots \\
0	0	0	\cdots	1
\end{matrix}
\right]
$$

\paragraph{微分方程中含输入函数导数项}%
\label{par:微分方程中含输入函数导数项}


\subsection{系统传递函数矩阵与状态空间表达式的相互转换}

\paragraph{系统传递函数}%
\label{par:系统传递函数}

\subsubsection{直接分解法}%
\label{ssub:直接分解法}

$\ldots$

\subsubsection{根据传递函数的分布情况建立系统状态空间表达式}

\paragraph{传递函数极点互异}%
\label{par:传递函数极点互异}



\end{document}
