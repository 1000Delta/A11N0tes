\documentclass[11pt]{book}
\usepackage{fontspec, xunicode, xltxtra}
\usepackage[tmargin=1in,bmargin=1in,lmargin=1.25in,rmargin=1.25in]{geometry}
\usepackage{ctex}
\usepackage{xcolor}
\usepackage{makeidx}

\begin{document}
    \begin{titlepage}
        \title{机器人学基础}
        \author{DX}
        \maketitle
    \end{titlepage}
    \tableofcontents

    \chapter{绪论}

    \chapter{数学基础}

    \chapter{机器人运动学}

    \section{机器人运动方程的表示}

    \section{机器人运动方程的求解}

    \subsection{逆运动学求解的一般问题}

    \subsubsection{解的存在性}

    \subsubsection{多解性问题}


    机器人系统在执行操控时只能选择一组解
    
    - 最短行程解
    - 较长行程解
    
    \subsubsection{逆运动学的求解方法}

    \subsection{逆运动学的代数解法和几何解法}
    
    \subsubsection{代数解法}


    \subsubsection{几何解法}

    \subsection{逆运动学的其他解法}

    \subsubsection{欧拉变换解法}

    欧拉方程表示运动姿态 P37

    公式(3.48)

    公式 (3.49) - (3.57) 
    不可使用反三角函数直接求解(未知数周期性)

    P51

    \paragraph{概括}

    如果已知一个表示任意旋转的齐次变换,那么就能确定其等价欧拉角:

    \begin{equation}
        \left\{
            \begin{array}{ll}
               \phi = atan2(a_y, a_x), \phi = \phi + 180 ^ \circ \\
               \theta = atan2(c\phi a_x + s \phi a_y, a_x) \\
               \psi = atan2(-s \phi n_x + c \phi n_y, -s \phi o_x + c \phi o_y)
            \end{array}
        \right.
    \end{equation}
\end{document}