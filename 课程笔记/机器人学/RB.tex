\documentclass[11pt]{book}
\usepackage{fontspec, xunicode, xltxtra}
\usepackage[tmargin=1in,bmargin=1in,lmargin=1.25in,rmargin=1.25in]{geometry}
\usepackage{ctex}
\usepackage{fontenc}
\usepackage{amsmath}

\begin{document}
\chapter{Robotics运动学}

\section{机器人运动方程的求解}

\section{机器人运动分析和综合举例}

以PUMA600为例

\section{机器人的雅可比公式}

\subsection{机器人的微分运动}

\subsubsection{微分平移与旋转}%
\label{ssub:wei_fen_ping_yi_yu_xuan_zhuan_}

基坐标系中左乘,运动坐标系中右乘

基坐标系:

$$
\begin{array}{ll}
	T + dT &= Trans(d_x, d_y, d_z)Rot(f,d \theta)T \\
	dT &= [Trans(d_x, d_y, d_z)Rot(f,d \theta)T - I]T
\end{array}
$$

坐标系 ${T}$:

$$
\begin{array}{ll}
	T + dT &= TTrans(d_x, d_y, d_z)Rot(f, d \theta) \\
	dT &= T[Trans(d_x, d_y, d_z)Rot(f, d \theta) - I]
\end{array}
$$

	当微分运动是对基系进行时,我们规定它为 $\triangle$ ;而运动是相对于坐标系 ${T}$ 来进行时,记为 ${}^T \triangle$ 。


\subsubsection{微分运动的等价变换}%
\label{ssub:wei_fen_yun_dong_de_deng_jie_bian_huan_}

\paragraph{目的}%
\label{par:mu_de_}

把一个坐标系内的位子变换到另一坐标系内

由 $dT = \Delta T = T {}^T \Delta$ 可得 ${}^T \Delta = T^{-1} \Delta T$

\subparagraph{向量叉乘}%
\label{par:xiang_liang_cha_cheng_}

$$
	\vec a \times \vec b = 
	\left|
		\begin{matrix}
			\vec i & \vec j & \vec k \\
			x_a & y_a & z_a \\
			x_b & y_b & c_b
		\end{matrix}
	\right|
	= (y_a c_b - z_a y_b) \vec i + (z_a x_b - x_a c_b) \vec j + (x_a y_b - y_a x_b) \vec k
$$
\end{document}
