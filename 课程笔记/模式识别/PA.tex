\documentclass[11pt]{article}
\usepackage{fontspec, xunicode, xltxtra}
\usepackage[tmargin=1in,bmargin=1in,lmargin=1.25in,rmargin=1.25in]{geometry}
\usepackage{ctex}
\usepackage{xcolor}

\begin{document}
\title{模式识别导论}
\author{DX}
\date{2019-09-21}

\section{绪论}

\section{聚类分析}

\subsection{距离聚类的概念}

\subsection{相似性测度和聚类准则}

\subsection{基于距离阈值的聚类算法}

\subsection{层次聚类法}

\subsection{动态聚类法}

聚类过程中,聚类中心位置或个数发生变化.

两种常用的算法:
\begin{itemize}
	\item K - 均值算法(或 C - 均值算法)
	\item 迭代自组织的数据分析算法
\end{itemize}

\subsubsection{K - 均值算法}%
\label{ssub:k_jun_zhi_suan_fa_}

基于使聚类准则函数最小化。

\paragraph{准则函数}%
\label{par:zhun_ze_han_shu_}

聚类集中每一样本点到该类中心的距离平方和。

对于第j个聚集类,准则函数定义为:
$$
   J_j	= \sum^{N_j}_{i=1} || X_i - Z_j ||^2, X_i \in S_j
$$

\paragraph{算法描述}%
\label{par:suan_fa_miao_shu_}

\paragraph{算法讨论}%
\label{par:suan_fa_tao_lun_}

\paragraph{聚类准则函数 $J_k$ 与 $K$ 的关系曲线}%
\label{par:ju_lei_zhun_ze_han_shu_j_k_yu_k_de_guan_xi_qu_xian_}

\subsubsection{迭代自组织的数据分析算法}%
\label{ssub:die_dai_zi_zu_zhi_de_shu_ju_fen_xi_suan_fa_}

迭代自组织的数据分析算法也常称为 ISODATA 算法 (Iterate Selft-Organizing Data Analysis Techniques Algorithm, ISODATA).

\paragraph{算法特点}%
\label{par:suan_fa_te_dian_}

\begin{itemize}
	\item 加入了试探性步骤,组成人机交互的结构;
	\item 可以通过类的自动合并与分裂得到较合理的类别数。
\end{itemize}

\paragraph{基本思路}%
\label{par:ji_ben_si_lu_}

\begin{enumerate}
	\item 选择初始值
	\item 按最邻近规则进行分类
	\item 聚类后的处理:计算各类中的距离函数等指标
	\item 判断结果是否符合要求,符合则结束,否则回到2
\end{enumerate}

\paragraph{算法描述}%
\label{par:suan_fa_miao_shu_}

P31

\paragraph{常用指标}%
\label{par:chang_yong_zhi_biao_}

各指标综合考虑

\begin{enumerate}
	\item 聚类中心之间的距离
	\item 诸聚类域中样本数目
	\item 诸聚类域中样本的标准差向量
\end{enumerate}

\section{判别函数及几何分类法}

\subsection{判别函数}

统计模式识别
$$
	\left\{
	\begin{array}{ll}
		& \mbox{聚类分析法(第二章)} \\
		& \mbox{判别函数法} 
		\left\{
		\begin{array}{ll}
			\left.
			& 
			\begin{array}{ll}
				\mbox{线性判别函数法} \\
				\mbox{非线性判别函数法} \\
			\end{array}
			\right\} => xx \\
			& \mbox{统计决策方法}
		\end{array}
		\right.
	\end{array}
	\right.
$$
\subsubsection{判别函数}%
\label{ssub:pan_bie_han_shu_}

\paragraph{定义}%
\label{par:ding_yi_}

直接用来对模式进行分类的准则函数。
\end{document}
