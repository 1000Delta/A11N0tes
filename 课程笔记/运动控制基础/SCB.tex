\documentclass[11pt]{book}
\usepackage{fontspec, xunicode, xltxtra}
\usepackage[tmargin=1in,bmargin=1in,lmargin=1.25in,rmargin=1.25in]{geometry}
\usepackage{ctex}
\usepackage{xcolor}
\usepackage{makeidx}

\begin{document}
\title{运动控制基础},\author{DX},\date{2019-09-19}
\maketitle
\tableofcontents
% \makeindex

直流电机的基本方程式

\chapter{磁路}
\index{磁路}

\chapter{直流电机}
\index{直流电机}

\section{直流电机的运行原理}
\index{直流电机!直流电机的运行原理}

\section{直流电机的基本方程式}%

\subsection{电动势平衡方程式}%
\label{sub:dian_dong_shi_ping_heng_fang_cheng_shi_}

\begin{equation}
	u_a = G_{af}i_f\Omega + R_ai_a + L_a \frac{di_a}{dt} 
\end{equation}

\begin{equation}
 u_f = R_fi_f + L_f \frac{di_f}{dt} 
\end{equation}
$$
	u_a = u_f = u
$$
式中
\begin{description}
	\item[$u$] --- 电源电压
	\item[$u_a$] --- 电枢绕组上的端电压
\end{description}
$\ldots$ 见P41

\subsection{转矩平衡方程式}%
\label{sub:zhuan_ju_ping_heng_fang_cheng_shi_}


$$
	T_{em} = T_2 + T_0 + J \frac{d\Omega}{dt} 
$$

转速越大转矩越小
$$
	P = F \cdot v
$$
\subsection{功率平衡}%
\label{sub:gong_lu_ping_heng_}

$$
	P = UI = U(I_f + I_a)
$$
\begin{description}
	\item[$P_{Cuf}$] 励磁损耗
	\item[$P_{Cua}$] 电枢铜耗
	\item[$P_c$] 电刷接触损耗
	\item[$P_{mech}$] 机械损耗
	\item[$P_{Fe}$] 铁心损耗
\end{description}

\section{直流电动机的运动特性}%

\subsection{概念}%
\label{sub:gai_nian_}

\subsection{表示式}%
\label{sub:biao_shi_shi_}

$n, T_{em}, \phi = f(I_a)$

\subsection{并励直流电动机的工作特性}

\subsubsection{转速特性}%
\label{ssub:zhuan_su_te_xing_}

$n = f(I_a)$

转速公式
\begin{equation}
	n = \frac{U}{C_e\Phi} - \frac{R_a}{C_e\Phi} I_a
\end{equation}
若不计电枢反应的去磁作用,可以认为 $\Phi$ 是一个与 $I_a$ 无关的常数。所以在 $U = U_n、I_f = I_{fN}$ 的条件下,转速特性可以表示为:
\begin{equation}
 n = n_0 - \beta'I_a
\end{equation}

\paragraph{例题}%
\label{par:li_ti_}

$$
	\mbox{例题1} \\
   U = C_e \Phi n + R_aI_a + 2\Delta U_cI_a
$$
当 $\Phi$ 减少 $10\%$ ,稳定时电枢电流
$$
	\mbox{例题2} \\
   U_N = C_e \Phi n + R_aI_a + 2\Delta U_cI_a \\
$$

\subsubsection{转矩特性}%
\label{ssub:zhuan_ju_te_xing_}

\paragraph{并励电动机的转矩特性}

$$
	T_e = C_T \Phi I_a = C'_T I_a
$$

\subsection{直流电机的换向}

\subsubsection{电抗电动势 $e_x$}%
\label{ssub:dian_kang_dian_dong_shi_e_x_}

\subsubsection{改善换向的方法}%
\label{ssub:gai_shan_huan_xiang_de_fang_fa_}

\paragraph{换向节}%
\label{par:huan_xiang_jie_}

\subsubsection{补偿绕组}%
\label{ssub:bu_chang_rao_zu_}


\end{document}
